\documentclass{article}

\usepackage{graphicx}

\begin{document}
\begin{center}
  \includegraphics[scale=0.1]{img/logo.png}
  
  \smallskip {\huge \textbf{Willow}}
  
  \smallskip {\Large Summer 2020 Proposal}
\end{center}

\begin{center}
  \textbf{Description}
\end{center}
\vspace{-10pt}
\noindent Willow is a web application that gives users the ability to create and share truth trees.
Truth trees are a common method of solving arguments and determining satisfiability in propositional logic.
The application is launched as a Node.js web server, using various libraries such as Express and Vue.js, as well as a PostgreSQL database to manage user accounts and other data.
The mission statement of Willow is to assist instructors in educating students about the purpose and usage of truth trees.
Willow takes a unique approach to the layout of truth trees; while truth trees are mostly visualized in a typical ``tree-like'' structure, Willow visualizes the tree similar to a file browser; branches can be expanded and collapsed at will in order to improve usability.

\begin{center}
  \textbf{Goals}
\end{center}
\vspace{-10pt}
\noindent Currently, Willow supports basic propositional logic truth trees, real-time validation, saving to a file, and opening a file.
In terms of improving functionality, the long-term goals for the project are to support first-order logic, add additional rules for decomposition, and emulate inference rules using truth trees.
Other quality-of-life improvements include accounts for saving files on the cloud, auto-grading, instructor accounts, and giving assignments (preset truth trees) to students.

\begin{center}
  \textbf{Timeline}
\end{center}
\begin{itemize}
  \item \textbf{June 14\textsuperscript{th} -- June 27\textsuperscript{th}}: Set up environment and refactor data structures used to represent truth tree from a traditional tree structure to a flattened ``identifier-based'' structure to improve efficiency of validator
  \item \textbf{June 28\textsuperscript{th} -- July 11\textsuperscript{th}}: Research secure methods of authentication and implement accounts (student and instructor) and assignments
  \item \textbf{July 12\textsuperscript{th} -- July 25\textsuperscript{th}}: Add alternative decomposition rules and support for emulating inference rules within truth trees
  \item \textbf{July 26\textsuperscript{th} -- August 8\textsuperscript{th}}: Expand validator to support first-order logic
  \item \textbf{August 9\textsuperscript{th} -- August 22\textsuperscript{nd}}: Finish implementing miscellaneous features (will be added to GitHub issue tracker throughout the semester) and prepare for presentation
\end{itemize}

\begin{center}
  \textbf{Contacts}
\end{center}
\begin{itemize}
  \item \textbf{Project Manager}: Connor Roizman (roizmc@rpi.edu)
\end{itemize}

\end{document}